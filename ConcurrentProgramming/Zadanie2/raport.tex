%%%%%%%%%%%%%%%%%%%%%%%%%%%%%%%%%%%%%%%%%
% University/School Laboratory Report
% LaTeX Template
% Version 3.1 (25/3/14)
%
% This template has been downloaded from:
% http://www.LaTeXTemplates.com
%
% Original author:
% Linux and Unix Users Group at Virginia Tech Wiki 
% (https://vtluug.org/wiki/Example_LaTeX_chem_lab_report)
%
% License:
% CC BY-NC-SA 3.0 (http://creativecommons.org/licenses/by-nc-sa/3.0/)
%
%%%%%%%%%%%%%%%%%%%%%%%%%%%%%%%%%%%%%%%%%

%----------------------------------------------------------------------------------------
%	PACKAGES AND DOCUMENT CONFIGURATIONS
%----------------------------------------------------------------------------------------

\documentclass{article}

\usepackage{pgfplots}
\usepackage{hyperref}
\usepackage{polski}
\usepackage[utf8]{inputenc}
\usepackage[polish]{babel}	
\usepackage[version=3]{mhchem} % Package for chemical equation typesetting
\usepackage{siunitx} % Provides the \SI{}{} and \si{} command for typesetting SI units
\usepackage{graphicx} % Required for the inclusion of images
\usepackage{natbib} % Required to change bibliography style to APA
\usepackage{amsmath} % Required for some math elements 

\setlength\parindent{0pt} % Removes all indentation from paragraphs

\renewcommand{\labelenumi}{\alph{enumi}.} % Make numbering in the enumerate environment by letter rather than number (e.g. section 6)

%\usepackage{times} % Uncomment to use the Times New Roman font

%----------------------------------------------------------------------------------------
%	DOCUMENT INFORMATION
%----------------------------------------------------------------------------------------

\title{Optymalizacja Algorytmu Brandesa \\ Programowanie Współbieżne } % Title

\author{Szymon \textsc{Pajzert}} % Author name

\date{\today} % Date for the report

\begin{document}

\maketitle % Insert the title, author and date

% If you wish to include an abstract, uncomment the lines below
% \begin{abstract}
% Abstract text
% \end{abstract}

%----------------------------------------------------------------------------------------
%	SECTION 1
%----------------------------------------------------------------------------------------

\section*{Cel}

Dokonanie optymalizacji współbieżnej implementacji algorytmu Brandesa, przedstawionego dokładniej \href{http://www.algo.uni-konstanz.de/publications/b-fabc-01.pdf}{tutaj}
 
%----------------------------------------------------------------------------------------
%	SECTION 2
%----------------------------------------------------------------------------------------

\section*{Dane testowe}
Do testowania wydajności użyte zostały dane wiki-vote-sort.txt powstałe z posortowania po pierwszej i drugiej kolumnie danych \href{http://snap.stanford.edu/data/wiki-Vote.html}{wiki-vote.txt}. \\

\begin{tabular}{ll}
Wierzchołki & 7115 \\
Krawędzie & 103689 \\
Wierzchołki w największym WCC & 7066 (0.993) \\
Krawędzie w największym WCC & 103663 (1.000) \\
Wierzchołki w największym SCC & 1300 (0.183) \\ 
Krawędzie w największym & 39456 (0.381) \\
Średni clustering coefficient & 0.1409 \\
Liczba trójkątów & 608389 \\
Średnica & 7
\end{tabular}

%----------------------------------------------------------------------------------------
%	SECTION 3
%----------------------------------------------------------------------------------------

\section*{Dokonane optymalizacje}
\begin{enumerate}
\item Nieużywanie mutexów - są drogie oraz istnieją inne metody synchronizacji - w tym wypadku atomics. 
Do incrementowania ich wartości użyłem \verb|compare_exchange_weak| - ale tylko jeśli wynik w \verb+delta+ jest dodatni. 
\item Zamiana numerów wierzchołków - pozwala na przetrzymywanie grafu w wektorze, dając mu stały lookup, przy stosunkowo znikomym narzucie wczytywania
\item Używanie \verb|unordered_map| - w testach wydajnościowych hashmapy dawały 3 razy większą wydajność w porównaniu z mapami opartymi o drzewa.
\item Użycie flagi \verb|gcc -O3|
\end{enumerate}

\section*{Benchmark}
Testy zostały wykonane na serwerze students.mimuw.edu.pl.

\begin{tikzpicture}
\begin{axis}[
    title={Czas działania bez -O3},
    xlabel={Liczba wątków},
    ylabel={Czas [s]},
    xmin=1, xmax=8,
    ymin=0, ymax=90,
    xtick={1,2,3,4,5, 6, 7, 8},
    ytick={0,10,20,40,60,80},
    legend pos=north east,
    ymajorgrids=true,
    grid style=dashed,
]
 
\addplot[
    color=blue,
    mark=square,
    ]
    coordinates {
    (1,77.7)(2,38)(3,26.6)(4,20.2)(5,15.9)(6,13.8)(7,12.1)(8,10.6)
    };
    \addlegendentry{Średni}
    
\addplot[
    color=red,
    mark=square,
    ]
    coordinates {
    (1, 81.849)(2, 40.771)(3, 28.105)(4, 21.704)(5, 17.344)(6, 14.296)(7, 12.292)(8, 10.772)
    };
    \addlegendentry{Najgorszy}
    
\addplot[
    color=green,
    mark=square,
    ]
    coordinates {
    (1, 74.606)(2, 36.301)(3, 24.721)(4, 17.805)(5, 13.622)(6, 13.573)(7, 11.523)(8, 10.432)
    };
    \addlegendentry{Najlepszy}
\end{axis}

\end{tikzpicture} \\
\begin{tikzpicture}

\begin{axis}[
    title={Przyspieszenie bez -O3},
    xlabel={Liczba wątków},
    ylabel={Wynik},
    xmin=0, xmax=8,
    ymin=0, ymax=8,
    xtick={1,2,3,4,5, 6, 7, 8},
    ytick={0,2,4,6,8},
    legend pos=north west,
    ymajorgrids=true,
    grid style=dashed,
]
 
\addplot[
    color=blue,
    mark=square,
    ]
    coordinates {
    (0, 0)(1, 1)
(2, 2.0466331129)
(3, 2.9229511586)
(4, 3.8557015369)
(5, 4.8829484986)
(6, 5.6317981027)
(7, 6.4537665194)
(8, 7.3270)
    };
    \addlegendentry{Średnie przyspieszenie}
    
\addplot[
    color=black,
    mark=square,
    dashed
    ]
    coordinates {
    (0, 0)(8, 8)
    };
    \addlegendentry{Idealne przyspieszenie}
    
\end{axis}
\end{tikzpicture} \\
\begin{tikzpicture}
\begin{axis}[
    title={Czas działania z -O3},
    xlabel={Liczba wątków},
    ylabel={Czas [s]},
    xmin=1, xmax=8,
    ymin=0, ymax=40,
    xtick={1,2,3,4,5, 6, 7, 8},
    ytick={0,10,20,30,40},
    legend pos=north east,
    ymajorgrids=true,
    grid style=dashed,
]
 
\addplot[
    color=blue,
    mark=square,
    ]
    coordinates {
(1, 32.6066666667)
(2, 16.607)
(3, 11.462)
(4, 9.122)
(5, 7.0143333333)
(6, 5.9993333333)
(7, 5.3576666667)
(8, 4.776)
    };
    \addlegendentry{Średni}
    
\addplot[
    color=red,
    mark=square,
    ]
    coordinates {
    (1, 34.007)
(2, 18.912)
(3, 11.674)
(4, 9.766)
(5, 7.339)
(6, 6.122)
(7, 5.378)
(8, 5.009)
    };
    \addlegendentry{Najgorszy}
    
\addplot[
    color=green,
    mark=square,
    ]
    coordinates {
(1, 31.109)
(2, 13.496)
(3, 11.081)
(4, 8.72)
(5, 6.661)
(6, 5.819)
(7, 5.341)
(8, 4.534)
};
    \addlegendentry{Najlepszy}
\end{axis}

\end{tikzpicture} \\
\begin{tikzpicture}

\begin{axis}[
    title={Przyspieszenie z -O3},
    xlabel={Liczba wątków},
    ylabel={Wynik},
    xmin=0, xmax=8,
    ymin=0, ymax=8,
    xtick={1,2,3,4,5, 6, 7, 8},
    ytick={0,2,4,6,8},
    legend pos=north west,
    ymajorgrids=true,
    grid style=dashed,
]
 
\addplot[
    color=blue,
    mark=square,
    ]
    coordinates {
(0, 0)
(1, 1)
(2, 1.9634290761)
(3, 2.8447624033)
(4, 3.5745085142)
(5, 4.6485767239)
(6, 5.4350483387)
(7, 6.0859827039)
(8, 6.8271915131)

    };
    \addlegendentry{Średnie przyspieszenie}
    
\addplot[
    color=black,
    mark=square,
    dashed
    ]
    coordinates {
    (0, 0)(8, 8)
    };
    \addlegendentry{Idealne przyspieszenie}
    
\end{axis}
\end{tikzpicture}

\end{document}
