\newpage
\section*{1 - 2}
Homeomorfizm pomiędzy
\begin{itemize}
  \item[1)] Sferą $S^n$
  \item[2)] $\mathbb{R} ^ {n+1} \setminus \{ 0 \} / \sim $ gdzie $ \mathbf{v} \sim  \mathbf{w} \iff \exists_{\lambda > 0}  \mathbf{v}=  \lambda \mathbf{w} $
\end{itemize}
\begin{itemize}

\item \textbf{(zwartość dziedziny)} Sfera $ S^n $ jest zbiorem zwartym: jest domkniętą i ograniczoną podprzestrzenią przestrzeni euklidesowej.\\

\item \textbf{(ciągłość $g$)} Weźmy przekształcenia
$$
f: S^n \rightarrow \mathbb{R}^{n+1} \backslash \{0\}
$$
$$
p: \mathbb{R}^{n+1} \backslash \{0\} \rightarrow \mathbb{R}^{n+1} \backslash \{0\} / \sim
$$
Przekształcenie $f$ jest włożeniem homeomorficznym, przekształcenie jest przekształceniem ilorazowym.
\\
Weźmy
$$ g: S^n \rightarrow \mathbb{R}^{n+1} \backslash \{0\} / \sim $$
Jest ono ciągłe, ponieważ jest złożeniem $f$ i $p$ - przekształcenia ilorazowego z przekształceniem ciągłym\\
\end{itemize}

Wprowadźmy nowy układ współrzędnych, naturalnie związany z homeomorfizmem 
$$
\eta : \mathbb{R}^n \rightarrow S^{n-1} \times \mathbb{R}
$$
Niech pierwsza współrzędna będzie promieniem sfery o środku w $0$, na której leży punkt, a druga położeniem na tej sferze. \\
Weźmy dwa różne elementy $ a, b \in X $, wtedy ich współrzędne to:
$$
a = (a_r, a_s)
$$
$$
b = (b_r, b_s)
$$
Przekształcenie $p$ jest utożsamieniem punktów o tej samej pierwszej współrzędnej.
\begin{itemize}
\item \textbf{(bijektywność $g$)} Reprezentantami warstw w $X$ mogą być więc wektory należące do sfery jednostkowej $S^n$ - to pokazuje, że $g$ jest bijekcją (każdej warstwie odpowiada jeden wektor na sferze jednostkowej).
\item \textbf{(hausdorffowość przeciwdziedziny)} Sfera $S^n$ jest Hausdorfa, można dla dwóch wektorów do niej należących można wybrać rozłączne otoczenia otwarte $U_a$ i $U_b$.
Zbiory
$$
V_a = p[U_a]
$$
$$
V_b = p[V_b]
$$
są rozłączne w $X$. Załóżmy że nie. Wtedy istnieją $ \alpha \in U_a, \beta \in U_b $ takie, że $ p(\alpha) = p(\beta) $, czyli $\alpha * t = \beta$ dla jakiegoś $t > 0$, ale oba należą do sfery jednostkowej, więc sprzeczność.\\
Pozostaje jedynie sprawdzić, że $V_a$ i $V_b$ są otwarte w $X$.

Weźmy $ W = p^{-1}[V_a] $ oraz $ x = (1, x_s) \in W $. Zrzutujmy $W$ na sferę o promieniu $1$ i weźmy otwartą kulę wokół punktu $x$ na tej sferze zawierającą się w rzucie $W$ o promieniu $r_x$ - można to zrobić, bo ten rzut jest otwarty. \\
Weźmy teraz kulę o promieniu $min(1, r_x)$ - jest ona zawarta w zbiorze $W$, bo punkty do niej należące po zrzutowaniu na sferę $S^n$ zawierają się w rzucie $W$. Wobec tego $W$ jest otwarty i przestrzeń  $\mathbb{R}^{n+1} \backslash \{0\} / \sim$  jest przestrzenią Hausdorffa.
\end{itemize}

Wiemy zatem, że $g$ jest ciągłą bijekcją z przestrzeni zwartej w przestrzeń Hausdorffa, więc jest homeomorfizmem.
