% {theorem}{Twierdzenie}[section]
% *{condition}{Kryterium}
% {corollary}{Wniosek}[theorem]
% {lemma}{Lemat}[theorem]
% {statement}{Stwierdzenie}[section]
% {problem}{Problem}[section]
% {question}{Pytanie}[section]
% *{definition}{Definicja}
% *{note}{Uwaga}

\section{Metody całkowania}
    \subsection{Funkcje wielomianowe}
        \begin{enumerate}
            \item Rozłożenie wielomianu na funckje wymierne proste
            \item Scałkowanie tych funkcji
        \end{enumerate}
        
    \subsection{Podstawienie uniwersalne}
        \begin{align*}
            t = tg \tfrac{x}{2} & &
            \sin x = \frac{2t}{1 + t^2} \\
            \cos x = \frac{1 - t^2}{1 + t^2} & &
            dx = \frac{2dt}{1 + t^2}
        \end{align*}
        
        
\section{Całka Oznaczona}
    \begin{definition}[Całka Newtona]
    Dla $F$ będącego funkcją pierwotną $f$ definiujemy:
        \begin{equation*}
            \int_a^b f(x) dx := F(a) - F(b)
        \end{equation*}
    \end{definition}
    
    \begin{theorem}[monotoniczność całki]
        \[ \forall_x f(x) \leq g(x) \implies \int_a^b f(x) dx \leq \int_a^b g(x) dx \]
    \end{theorem}
    
    \begin{theorem}[wartość średnia]
        \[ \exists y \in (a, b)  \int_a^b f(x) dx = f(y)(b-a) \]
    \end{theorem}