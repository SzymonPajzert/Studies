\section{Kryteria zbieżności}

\begin{condition}[konieczne]
    Jeśli szereg $\sum a_n$ jest zbieżny to $\lim a_n = 0$
\end{condition}

\begin{condition}[Cauchy'ego]
    $\sum a_n$ jest zbieżny $\iff$ gdy $\forall_{\epsilon}$ $\exists_{N}$ takie że $\forall_{n,m}$ takiego że  $n > m > N$ zachodzi:
    \begin{equation*}
        \sum \limits_{i=m}^{n} a_i < \epsilon
    \end{equation*}
\end{condition}

\begin{condition}[porównawcze]
    $\forall_{n > N_0}$ $|a_n| < c_n$ $\land$ $\sum c_n$ zbieżny $\implies$ $\sum |a_n|$ zbieżny.
\end{condition}

\begin{condition}[kondensacyjne]
    $a_n$ są nierosnące i nieujemne to:
    \begin{equation*}
        \sum a_n \text{ zbieżny} \iff \sum 2^n \cdot a_{2^n} \text{ zbieżny}
    \end{equation*}
\end{condition}

\begin{condition}[Cauchy'ego]
    $\limsup \sqrt[n]{|a_n|} < 1 \implies \sum a_n \text{ zbieżny}$
\end{condition}
    
\begin{theorem}[Cauchy'ego-Hadammonda]
    Niech $\mu = \limsup \sqrt[n]{|a_n|}$. Wtedy $\sum a_n \cdot x^n$ ma promień zbieżności $R = \mu^{-1}$
\end{theorem}

\begin{condition}[D'Alamberta]
    $\limsup \sqrt[n]{|a_n|} < 1 \implies \sum a_n \text{ zbieżny}$
\end{condition}

\begin{corollary}
    Kryterium D'Alamberta jest słabsze niż kryterium Cauchy'ego.
\end{corollary}

\begin{condition}[Leibniza]
    $a_n$ są malejące i zbieżne do zera, nieujemne. Wtedy $\sum (-1)^n a_n$ jest zbieżne. 
\end{condition}

\begin{theorem}[Mertensa]
    Jeśli $\sum a_n$ jest bezwzględnie zbieżny i $\sum b_n$ jest zbieżny to ich iloczyn Cauchy'ego jest zbieżny do iloczynu ich granic.
\end{theorem}

\begin{theorem}[Abela]
    Jeśli iloczyn Cauchy'ego szeregów jest zbieżny, to jego granica równa się iloczynowi granic tych szeregów.
\end{theorem}

\begin{theorem}[Riemanna]
    Niech $\sum a_n$ będzie szeregiem zbieżnym warunkowo. Wówczas istnieje przestawienie porządku $\sum a'_n$ że dla $\alpha \leq \beta$:
    \begin{align*}
        \liminf \sum a'_n = \alpha \\ \limsup \sum a'_n = \beta
    \end{align*}
\end{theorem}

\begin{corollary}
    Szereg bezwzględnie zbieżny nie zmienia swojej granicy niezależnie od przestawienia wyrazów.fa
\end{corollary}

\begin{condition}[Abela]
Jeśli zachodzi:
\begin{enumerate}
    \item Dla dowolnego $x \in A$ ciąg $f_n(x)$ jest monotoniczny
    \item $f_n(x)$ jest jednostajnie ograniczony na A.
    \item $\sum g_n(x)$ jednostajnie zbieżny
\end{enumerate}
To $\sum f_n(x) g_n(x)$ jednostajnie zbieżny.
\end{condition}

\begin{condition}[Dirichleta]
Jeśli zachodzi:
\begin{enumerate}
    \item $g_n(x)$ monotonicznie zbiega do 0.
    \item $\sum | f_n(x) |$ jest jednostajnie ograniczony na A.
\end{enumerate}
To $\sum f_n(x) g_n(x)$ jednostajnie zbieżny.

\begin{condition}[Weierstrassa]
    Jeśli ciąg wartości bezwzględnych funkcji ogranicza się na całym przedziale pewnym ciągiem liczb, którego szereg jest bezwzględnie zbieżny, to szereg tych funkcji jest jednostajnie zbieżny na tym przedziale.
\end{condition}

\end{condition}
