\section{Przestrzenie sprzężone - dwoistość}

% {theorem}{Twierdzenie}[section]
% *{condition}{Kryterium}
% {corollary}{Wniosek}[theorem]
% {lemma}{Lemat}[theorem]
% {statement}{Stwierdzenie}[section]
% {problem}{Problem}[section]
% {question}{Pytanie}[section]
% *{definition}{Definicja}
% *{note}{Uwaga}

\begin{statement}
    Istnieje kanoniczny izomorfizm pomiędzy $\epsilon_L : L \to L**$ zadany wzorem:
    \begin{equation*}
        \epsilon_L(l)(f) = f(l)
    \end{equation*}
\end{statement}

\begin{definition}[wektor sprzężony]
    Dla $\alpha \in V$ definiujemy $\alpha^* \in V^*$, takie że:
    \begin{equation*}
        \alpha^* (\beta) = repr(\alpha^*)^{T} \cdot  repr(\beta)
    \end{equation*}
    Gdzie $repr$ to reprezentacja wektora jako wektora współrzędnych odpowiednio w bazie sprzężonej i zwykłej.
\end{definition}

\begin{definition}[odwzorowanie sprzężone]
    Dla $f \in \text{End}(L, M)$ definiujemy $f^* \in \text{End}(M^*, L^*)$, takie, że:
    \begin{equation*}
        (f^*(m^*))(l) = m^*(f(l))
    \end{equation*}
    $f^*$ jest jednoznacznie wyznaczony.
\end{definition}

\begin{statement}
    Operacja sprzężenia operatorów liniowych jest liniowa, kontrawariantna a jej kwadrat jest 
    identycznością z dokładnością do izomorfizmu naturalnego.
\end{statement}

\begin{statement}
    Sprzężenie wektora danej przestrzeni to transpozycja. W szczególności dla wektora należącego do $\text{End}(L, M)$, macierz jego sprzężenia w $\text{End}(M^*, L^*)$ jest transpozycją macierzy tego endomorfizmu.
\end{statement}

\begin{definition}[dopełnienie ortogonalne przestrzeni]
    Dla $M \leq L$ definiujemy $M^{\perp} = \set{ \alpha^* \in L^* \mid \forall_{\beta \in M} \alpha^*(\beta) = 0 } \leq L^*$
\end{definition}

\begin{statement}
    Istnieje izomorfizm kanoniczny $L* / M^{\perp} \cong M*$
\end{statement}
\begin{statement}
    Jeśli endomorfizm jest monomorfizem to jego sprzężenie jest endomorfizmem i na odwrót.
\end{statement}