\section{Ortogonalizacja Grama-Schmidta}

% {theorem}{Twierdzenie}[section]
% *{condition}{Kryterium}
% {corollary}{Wniosek}[theorem]
% {lemma}{Lemat}[theorem]
% {statement}{Stwierdzenie}[section]
% {problem}{Problem}[section]
% {question}{Pytanie}[section]
% *{definition}{Definicja}
% *{note}{Uwaga}

\begin{theorem}[Ortogonalizacja Grama-Schmidta]
    Niech $\beta_i$ będzie bazą przestrzni unitarnej $V$. Wówczas układ wektorów $\gamma_i$ dany wzorem:
    \begin{equation*}
        \gamma_n = \beta_n - \sum_{i=1}^{n-1} \frac{\beta_n, \gamma_i}{\langle \gamma_i, \gamma_i \rangle} \gamma_i 
    \end{equation*}
    Jest bazą ortogonalną (którą łatwo zortonormalizować).
\end{theorem}