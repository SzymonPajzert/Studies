\section{Podstawy}

\begin{lemma}{Kuratowskiego-Zorna}
    Jeśli każdy łańcuch ma ograniczenie górne, to ograniczenia górne są elementami maksymalnymi.
\end{lemma}

\begin{statement}
    $H$ jest $f$ niezmiennicza $\implies$ $T_H$ jest $T_f$ niezmiennicza. Implikacja zachodzi tylko w jedną stronę - np. translacja.
\end{statement}

\begin{statement}[sztuczka]
    Każda baza afiniczna jest sobie równoważna - dla ułatwienia obliczeń można przechodzić do bardziej kanonicznych.
\end{statement}

\begin{statement}
    Zbiór automorfizmów zachowujących daną figurę wypukłą tworzy grupę. Szukanie tej grupy to głównie eliminowanie automorfizmów, nie zachowujących pewnych niezmienników - naprzeciwległość wierzchołków w kwadracie. Automorfizmy te definiowane są na wierzchołkach.
\end{statement}

\begin{statement}
    W wielościanach dualnych każdy wierzchołek zadaje ścianę, a każda ściana zadaje wierzchołek.
\end{statement}