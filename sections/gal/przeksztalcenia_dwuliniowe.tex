\section{Przekształcenia dwuliniowe}

% {theorem}{Twierdzenie}[section]
% *{condition}{Kryterium}
% {corollary}{Wniosek}[theorem]
% {lemma}{Lemat}[theorem]
% {statement}{Stwierdzenie}[section]
% {problem}{Problem}[section]
% {question}{Pytanie}[section]
% *{definition}{Definicja}
% *{note}{Uwaga}

\begin{definition}
    Przekształcenie $f : V_{1} \times \dots \times V_{n} \to W$ jest $n$-liniowe jeśli dla ustalonego $i$-tego argumentu, przekształcenie jest $(n-1)$-liniowe.
\end{definition}

\begin{definition}[iloczyn tensorowy]
    Niech $V$ oraz $W$ będą przestrzeniami wektorowymi nad tym samym ciałem. Iloczyn tensorowy to $V \otimes W$ wraz z przekształceniem dwuliniowym $\mu : V \times W \to V \otimes W$ spełniającym następującą własność uniwersalną:
    \begin{equation*}
        \forall \text{\,dwuliniowe\,}f: V \times W \to X \exists ! u \in End(V \otimes W, X) : f = \mu u
    \end{equation*}
\end{definition}

\theorem Iloczyn tensorowy istnieje dla dowolnych $V$ i $W$. Definiujemy $R$ jako zbiór zawierający:
\begin{align*}
    R & \ni (\alpha_1 + \alpha_2, \beta) - (\alpha_1, \beta) - (\alpha_2, \beta) \\
    R & \ni (\alpha, \beta_1 + \beta_2) - (\alpha, \beta_1) - (\alpha, \beta_2) \\
    R & \ni c(\alpha, \beta) - (c\alpha, \beta) \\
     R & \ni c(\alpha, \beta) - (\alpha, c\beta)
\end{align*}
$V \otimes W$ definiujemy jako iloraz $V \times X$ oraz $\text{lin }R$

\begin{corollary}Jeśli $V$ ma bazę $\set{\alpha_i}$ a $W$ ma bazę $\set{\beta_j}$ to bazą $V \otimes W$ jest $\set{\alpha_i \otimes \beta_j}$
\end{corollary}

\begin{example}
\[ \text{Hom } (V, W) \cong V^* \otimes W \]
\end{example}


\begin{corollary} $ K \otimes V \cong V $ \end{corollary}
\begin{corollary} $ V \otimes W \cong W \otimes V $ \end{corollary}
\begin{corollary} $ (V \otimes W) \otimes X \cong V \otimes (W \otimes X) $ \end{corollary}
\begin{corollary} $ (V \oplus W) \otimes X \cong (V \otimes X) \oplus (W \otimes X) $ \end{corollary}
\begin{corollary}
Jeśli $V$ oraz $W$ są skończenie wymiarowe to:
\[ (V \otimes W)^* \cong V^* \otimes W^* \]
\end{corollary}

\begin{statement}
    $\Phi : V^* \otimes W \to \text{Hom}(V, W)$ dane wzorem:
    \[ \Phi(f \otimes \beta)(\alpha) = f(\alpha) \cdot \beta \]
    jest monomorfizmem. Jeśli $V$ oraz $W$ są skończenie wymiarowe, to jest to izomorfizm.
\end{statement}