\section{Iloczyn skalarny}
Założenie dla wszystkich poniższych stwierdzeń: $K = \mathbb{R}$

% {theorem}{Twierdzenie}[section]
% *{condition}{Kryterium}
% {corollary}{Wniosek}[theorem]
% {lemma}{Lemat}[theorem]
% {statement}{Stwierdzenie}[section]
% {problem}{Problem}[section]
% {question}{Pytanie}[section]
% *{definition}{Definicja}
% *{note}{Uwaga}

\begin{definition}[przekształcenie symetrzyczne]
    $ \forall_{\alpha, \beta} \langle \alpha, \beta \rangle = \langle \beta, \alpha \rangle $
\end{definition}

\begin{definition}[przekształćenie dodatnio określone] 
    \[ \langle \alpha, \alpha \rangle \geq 0 \land \langle \alpha, \alpha \rangle = 0 \implies \alpha = 0 \]
\end{definition}

\begin{definition}[iloczyn skalarny]
    Symetryczne i dodatnio określone przekształcenie dwuliniowe. 
\end{definition}

\begin{theorem}
    Każdy iloczyn skalarny nad $\mathbb{R}^n$ ma postać:
    \begin{align*}
        \langle (x_1, ..., x_n), (y_1, ..., y_n) \rangle = \sum_i \sum_j c_{ij} x_i y_j & & \text{gdzie }c_{ij} = c_{ji}
    \end{align*}
\end{theorem}

\begin{definition}[Przestrzeń unitarna]
    Przestrzeń liniowa ze zdefiniowanym iloczynem skalarnym.
\end{definition}

\begin{definition}[miara]
    $ || \alpha || = \sqrt{\langle \alpha, \alpha \rangle} $
\end{definition}

\begin{definition}[wektory ortogonalne]
    $ \iff \langle  \alpha, \beta \rangle = 0 $ 
\end{definition}

\begin{corollary}[trójkąt]
    $ || \alpha + \beta || \leq || \alpha || + || \beta || $
\end{corollary}

\begin{corollary}[Pitagoras]
    $  \langle  \alpha, \beta \rangle = 0 \iff || \alpha + \beta ||^2 = || \alpha ||^2 + || \beta ||^2 $
\end{corollary}

\begin{corollary}[Schwarz]
    $ \langle \alpha, \beta \rangle \leq ||\alpha|| \cdot || \beta || $
\end{corollary}

\begin{proof}
    \begin{align*}
        \langle t \alpha + \beta, t \alpha + \beta \rangle \geq 0 \\
        \langle t \alpha, t \alpha \rangle + \langle t \alpha, \beta \rangle + \langle \beta, t \alpha \rangle + \langle  \beta, \beta \rangle \geq 0 \\
        t^2\langle \alpha, \alpha \rangle + 2t\langle \alpha, \beta \rangle + \langle  \beta, \beta \rangle \geq 0
    \end{align*}
    Jest co najwyżej jedno $t$ spełniające tę nierówność, więc:
    \begin{align*}
        (2\langle \alpha, \beta \rangle)^2 - 4 \langle \alpha, \alpha \rangle \cdot \langle  \beta, \beta \rangle \leq 0 \\
        \langle \alpha, \beta \rangle \leq ||\alpha|| \cdot || \beta ||
    \end{align*}
\end{proof}

\begin{definition}[kąt] Liczba $\theta \in [0, \pi]$, że:
    \[ \theta  = \arccos \frac{\langle \alpha , \beta \rangle }{||\alpha|| \cdot || \beta ||} \]
\end{definition}

\begin{definition}[baza ortogonalna] baza złożona z wektorów prostopadłych.\end{definition}
\begin{definition}[baza ortonormalna] baza ortogonalna złożona z wektorów długości 1. \end{definition}
\begin{statement}
    Jeśli $\beta_i$ jest bazą ortonormalną to zachodzi: \[\beta_i^*(\alpha) = <\alpha, \beta_i>\]
\end{statement}