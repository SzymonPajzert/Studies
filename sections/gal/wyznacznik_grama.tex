\section{Wyznacznik Grama}

% {theorem}{Twierdzenie}[section]
% *{condition}{Kryterium}
% {corollary}{Wniosek}[theorem]
% {lemma}{Lemat}[theorem]
% {statement}{Stwierdzenie}[section]
% {problem}{Problem}[section]
% {question}{Pytanie}[section]
% *{definition}{Definicja}
% *{note}{Uwaga}


\begin{definition}[Macierz Grama] 
    DLa danego układu wektorów $\alpha_i$ to taka macierz $G(\alpha_1, ..., \alpha_n) = g_{ij}$, że zachodzi:
    \[ g_{ij} = <\alpha_i, \alpha_j> \]
\end{definition}

\begin{definition}[wyznacznik Grama] Wyznacznik macierzy Grama. \end{definition}

\begin{corollary}
    Jeśli $\alpha_i$ jest baza ortonormalną, to jej wyznacznik Grama wynosi 1.
\end{corollary}

\begin{statement}
    Niech $\beta_j$ będzie bazą ortogonalną, a $\alpha_i$ dowolnym układem wektorów przestrzeni unitarnej przez nią rozpiętej. Wtedy:
    \begin{align*}
        G(\alpha_1, ..., \alpha_k) &= A^TA\text{, gdzie} \\
        A &= a_{ij}\text{, gdzie} \\
        a_{ji} &= \langle \beta_j, \alpha_i \rangle
    \end{align*}
\end{statement}

\begin{statement}
    Niech $\beta_i$ będzie bazą $V$. $G(\beta_1, ..., \beta_n$ jest diagonalna $\iff$ baza ta jest ortogonalna.
\end{statement}

\begin{statement}
    Dla dwóch baz $\beta_i$ $\gamma_i$ zachodzi:
    \begin{equation*}
        G(\beta_1, ..., \beta_n) = (M(id)_B^Y)^T G(\gamma_1, ..., \gamma_n) M(id)_B^Y
    \end{equation*}
\end{statement}

