\documentclass{article}

\usepackage{pgfplots}
\usepackage{hyperref}
\usepackage{polski}
\usepackage[utf8]{inputenc}
\usepackage[polish]{babel}	
\usepackage[version=3]{mhchem} % Package for chemical equation typesetting
\usepackage{siunitx} % Provides the \SI{}{} and \si{} command for typesetting SI units
\usepackage{graphicx} % Required for the inclusion of images
\usepackage{natbib} % Required to change bibliography style to APA
\usepackage{amsmath} % Required for some math elements 

\setlength\parindent{0pt} % Removes all indentation from paragraphs

\renewcommand{\labelenumi}{\alph{enumi}.} % Make numbering in the enumerate environment by letter rather than number (e.g. section 6)

%\usepackage{times} % Uncomment to use the Times New Roman font

%----------------------------------------------------------------------------------------
%	DOCUMENT INFORMATION
%----------------------------------------------------------------------------------------

\title{ Oświetlenie miejskich rowerów } % Title

\author{Szymon \textsc{Pajzert}} % Author name
\date{}

\begin{document}

\maketitle % Insert the title, author and date

% If you wish to include an abstract, uncomment the lines below
% \begin{abstract}
% Abstract text
% \end{abstract}

%----------------------------------------------------------------------------------------
%	SECTION 1
%----------------------------------------------------------------------------------------

Z agregacji danych otrzymaliśmy, że w dzień odbyło się 48,142,817 a w nocy 18,495,221 początków lub końców przejazdów. Zakładająć że ułamek $p$ (z przedziału 0 do 1) przejazdów nadal będzie możliwych oraz zakładając globalny przyrost uczestników programu o 30\% otrzymamy następujące wyniki. Okazuje się, że operacja ta przyniesie nam pozytywne skutki tylko gdy przynajmniej 20\% przejazdów będzie mogło się odbyć po ciemku. Otrzymamy 10\% zmianę dopiero przy 30\%. Jeśli więc naszym priorytetem jest zwiększenie przejazdów, usunięcie oświetlenia raczej pozytywnie na ten aspekt nie wpłynie. \\ 

\begin{tikzpicture}
\begin{axis}[
	legend pos = south east, 
    axis lines = left,
    xlabel = $p$,
    ylabel = Liczba przejazdów,
    ymin=50000000, ymax=75000000,
]
%Below the red parabola is defined
\addplot [
    domain=0:40, 
    samples=100, 
    color=red,
]
{(48142817 + (x/100) * 18495221)*1.30};
\addlegendentry{Usunięcie oświetlenia}
%Here the blue parabloa is defined
\addplot [
    domain=0:40, 
    samples=10, 
    color=blue,
    ]
    {48142817 + 18495221};
\addlegendentry{Aktualna sytuacja}
 
\end{axis}
\end{tikzpicture} \\

Poniżej znajduje się dokładniejszy opis mojej metodologi oraz użytych narzędzi.

\newpage
\section*{Metodologia}

W obliczeniach każdy przejazd jest liczony jako dwa przejazdy w jednym momencie - jego początek i koniec. Przejazdy są później akumulowane w pięciominutowe interwały, gdzie interwał 0 to pierwsze przejazdy odbywające się po zachodzie słońca lub ostatnie przed. Dajsze ujemne numery oznaczają numery w nocy, a dodatnie w dzień. Poniższy wykres dodatkowo wizualizuje udział tych interwałów w przejazdach. Do obliczenia wyników użyłem chmury BigQuery i dostępnej na niej możliwości dokonywania zapytań SQL. \\

\begin{tikzpicture}
\begin{axis}[
    title={Wykres przejazdów},
    xlabel={Przedział od zmiany dzień/noc},
    ylabel={Liczba przejazdów},
    xmin=-94, xmax=100,
    ymin=0, ymax=900000,
    xtick={-90,-50, 0, 50, 100},
    ytick={0,200000,400000,600000,800000},
    legend pos=north east,
    ymajorgrids=true,
    grid style=dashed,
]
 
\addplot[
    color=blue,
    smooth
    ]
    coordinates {
    (-94,18)
(-93,97)
(-92,209)
(-91,313)
(-90,441)
(-89,999)
(-88,2358)
(-87,3142)
(-86,3662)
(-85,4353)
(-84,5082)
(-83,6231)
(-82,7381)
(-81,8418)
(-80,9387)
(-79,10919)
(-78,12293)
(-77,13585)
(-76,14566)
(-75,16140)
(-74,18185)
(-73,20773)
(-72,23098)
(-71,25059)
(-70,27216)
(-69,30751)
(-68,34242)
(-67,38666)
(-66,43156)
(-65,46290)
(-64,49307)
(-63,52228)
(-62,55373)
(-61,59481)
(-60,64350)
(-59,69667)
(-58,75242)
(-57,81586)
(-56,88421)
(-55,96317)
(-54,103424)
(-53,110691)
(-52,118336)
(-51,126136)
(-50,135272)
(-49,145130)
(-48,156016)
(-47,166766)
(-46,177371)
(-45,190097)
(-44,202163)
(-43,215275)
(-42,229137)
(-41,240726)
(-40,250214)
(-39,258561)
(-38,267055)
(-37,275455)
(-36,284060)
(-35,291360)
(-34,298737)
(-33,305618)
(-32,311611)
(-31,317527)
(-30,323074)
(-29,330154)
(-28,336294)
(-27,341741)
(-26,347696)
(-25,355426)
(-24,360294)
(-23,366695)
(-22,372342)
(-21,378720)
(-20,383701)
(-19,387954)
(-18,392384)
(-17,395897)
(-16,399370)
(-15,402264)
(-14,405119)
(-13,407320)
(-12,408402)
(-11,409504)
(-10,410329)
(-9,409501)
(-8,405827)
(-7,401616)
(-6,396493)
(-5,389868)
(-4,385811)
(-3,383145)
(-2,380304)
(-1,379654)
(0,376928)
(1,375654)
(2,376148)
(3,376571)
(4,376647)
(5,378366)
(6,379664)
(7,382036)
(8,384584)
(9,388945)
(10,393208)
(11,399900)
(12,405106)
(13,409744)
(14,416187)
(15,422824)
(16,430036)
(17,435229)
(18,441267)
(19,448224)
(20,457177)
(21,463359)
(22,472256)
(23,479155)
(24,488601)
(25,497860)
(26,507271)
(27,517152)
(28,523976)
(29,533800)
(30,542032)
(31,549144)
(32,557833)
(33,568817)
(34,579425)
(35,589253)
(36,597085)
(37,609549)
(38,621212)
(39,630786)
(40,644126)
(41,656538)
(42,667802)
(43,679647)
(44,691838)
(45,706136)
(46,718016)
(47,729960)
(48,740638)
(49,749932)
(50,762467)
(51,770547)
(52,781944)
(53,789872)
(54,800402)
(55,795606)
(56,771595)
(57,763025)
(58,755316)
(59,749997)
(60,741347)
(61,736360)
(62,726192)
(63,719080)
(64,711761)
(65,702435)
(66,693013)
(67,680949)
(68,670808)
(69,658293)
(70,642607)
(71,624181)
(72,603309)
(73,582726)
(74,558308)
(75,537197)
(76,515385)
(77,491324)
(78,468860)
(79,445608)
(80,419312)
(81,392303)
(82,364129)
(83,338962)
(84,310792)
(85,280791)
(86,246722)
(87,209266)
(88,168205)
(89,115963)
(90,30796)
(91, 0)
    };    
\end{axis}

\end{tikzpicture} \\

\newpage
\section*{Zapytania SQL}
Do dokonania zapytań SQL, użyłem modelu ziemi jako kuli krążącej po okręgu wokół słońca. Do obliczeń użyłem modelu \href{https://en.wikipedia.org/wiki/Sunrise_equation}{sunrise equation}.
\begin{verbatim}
#standardSQL
-- Number of days in a year
CREATE TEMPORARY FUNCTION days(d DATE)
RETURNS INT64 AS (
    CASE
    WHEN 
        (MOD(EXTRACT(YEAR FROM d),4) = 0 
            AND MOD((EXTRACT(YEAR FROM d)),100) != 0)
        OR MOD((EXTRACT(YEAR FROM d)),400) = 0 
        THEN 365
    ELSE 366
    END
);

-- Convert angle to radians
CREATE TEMPORARY FUNCTION rad(ang FLOAT64)
RETURNS FLOAT64 AS (3.141592 * ang / 180);

-- Calculate declination of the sun
CREATE TEMPORARY FUNCTION delta(d DATE)
RETURNS FLOAT64 AS (
    rad(-23.45)
    * cos(rad((360 / days(d))
    * (EXTRACT(DAYOFYEAR FROM d) + 10)))
);

-- Returns latitude of the New York
CREATE TEMPORARY FUNCTION psi()
RETURNS FLOAT64 AS (rad(40.748433));

-- Calculates hour angle of the earth from sun eguation
CREATE TEMPORARY FUNCTION sun_equation(d DATE)
RETURNS FLOAT64 AS (ACOS(- TAN(psi()) * TAN(delta(d))));

-- Converts time of the day to float in range from [1, to 24)
CREATE TEMPORARY FUNCTION time_to_float(t TIME)
RETURNS FLOAT64 AS (
    EXTRACT(HOUR FROM t) 
    + EXTRACT(MINUTE FROM t) / 60 + EXTRACT(SECOND FROM t) / 3600
);

-- Calculates noon time due to daylight saving
CREATE TEMPORARY FUNCTION noon(d DATE)
RETURNS FLOAT64 AS (
    CASE
        WHEN EXTRACT(MONTH FROM d) < 3 OR EXTRACT(MONTH FROM d) > 10 THEN 12
        ELSE 13
    END
);

-- Calculate length of half of a day
CREATE TEMPORARY FUNCTION halfday(d DATE)
RETURNS FLOAT64 AS (sun_equation(d) * 12 / 3.1415);

-- Calculate time of sunrise
CREATE TEMPORARY FUNCTION sunrise(d DATE)
RETURNS FLOAT64 AS (noon(d) - halfday(d));

-- Calculate time of sunset
CREATE TEMPORARY FUNCTION sunset(d DATE)
RETURNS FLOAT64 AS (noon(d) + halfday(d));

-- Return smallest distance of t from a and b,
-- positive when t in (a, b) 
-- negative otherwise
CREATE TEMPORARY FUNCTION closest(t FLOAT64, a FLOAT64, b FLOAT64)
RETURNS FLOAT64 AS (
    CASE
        WHEN t < a THEN t - a
        WHEN t > b THEN b - t
        ELSE 
            (CASE WHEN t - a < b - t THEN t - a ELSE b - t END)
    END
);
\end{verbatim}
\newpage
Oto pod zapytania użyte do dokonania akumulacji wyników.
\begin{verbatim}
WITH   
-- Extract times of start and stop
    times AS (
    SELECT * FROM (
        SELECT DATE(starttime) as d, starttime as timestamp, 
            time_to_float(TIME(starttime, "America/New_York")) as time 
        FROM `bigquery-public-data.new_york.citibike_trips`)
    UNION ALL (
        SELECT DATE(starttime) as d, stoptime as timestamp, 
            time_to_float(TIME(stoptime, "America/New_York")) as time 
        FROM `bigquery-public-data.new_york.citibike_trips`)),
  
-- Extract offset as number of five minutes intervals from sunset/sunrise of given start/stop  
    data AS (
    SELECT d, timestamp, time, 
        CAST(TRUNC(12 * closest(time, sunrise(d), sunset(d))) AS INT64) as offset
    FROM times),

-- Aggregate offsets
    offsets AS (
    SELECT count(*) as number, offset
    FROM data
    GROUP by offset),
  
-- Count given interval as daily or nightly
    gradation AS ( 
    SELECT offset, 
        CAST(offset > 0 AS INT64) * number as day, 
        CAST(offset <= 0 AS INT64) * number as night
    FROM offsets)
    
-- Accumulate results    
SELECT sum(day), sum(night) from gradation;
\end{verbatim}

\end{document}

