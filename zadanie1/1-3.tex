\section*{1-3}
Homeomorfizm pomiędzy
\begin{itemize}
  \item[1)] Sfera $S^n$
  \item[3)] $(\R^n)^+$ - sfera Riemanna
\end{itemize}
\begin{itemize}
\item \textbf{Sfera $S^n$ bez punktu jest homeomorficzna z $\mathbb{R}^n$}\\
Zanurzamy wszystko w $\mathbb{R}^{n+1}$, bierzemy sferę o środku w $(0, .. 0, 1)$ promieniu $1$ oraz hiperpłaszczyznę n-wymiarową taką, że ostatnia współrzędna jest równa $0$, tzn zbiór $\{(a_1, ... a_n, 0): a_i \in \mathbb{R}\}$.
Weźmy teraz rzut stereograficzny $h: S^n \backslash \{(0, ... 0, 1)\} \rightarrow \mathbb{R}^n$ takie, że $h(x) = y \iff x, y, (0, ... 0, 2)$ są współliniowe, czyli zapisując wzorem
$$
h((x_1, ... x_n)) = (\frac{x_1}{2-x_n}, \frac{x_2}{2-x_n}, .. \frac{x_{n-1}}{2-x_n}, 0)
$$
Funkcja ta jest ciągła, bo jest ciągła na każdej współrzędnej, z arytmetycznych własności funkcji ciągłych.
Jest różnowartościowa, bo prosta albo jest cała zawarta w hiperpłaszczyźnie (niemożliwe, bo przecina punkt nienależący do niej), albo przecina ją w jednym punkcie, albo jest z nią rozłączna (nie jest, bo przecina punkty o różnych współrzędnych $x_{n+1}$.
Można więc mówić o funkcji odwrotnej na obrazie $h[S^n]$ - jest ona dana wzorem
$$
h^{-1}((p_1, ... , p_n, 0)) = (p_1 * (2-S), p_2*(2-S), .. p_{n-1} * (2-S), S)
$$
$$
S = \frac{2*(p_1^2 + ... + p_n^2)}{1 + p_1^2 + ... + p_n^2}
$$
$h^{-1}$ jest różnowartościowa - widaćto ze wzoru, więc $h$ jest bijekcją.
Z drugiej strony, $h^{-1}$ jest ciągła (każda z funkcji współrzędnych jest ciągła), więc $h$ jest homeomorfizmem.

\item \textbf{Sfera $S^n$ jest zwarta}\\
Jest domkniętą i ograniczoną podprzestrzenią przestrzeni euklidesowej, więc jest zwarta.

\item \textbf{Dołożenie punktu jest uzwarceniem sfery bez punktu}\\
Sfera $S^n \backslash (0, 0, ... 1)$ jest pozbiorem gęstym w $S^n$ oraz jest zanurzone w niej homeomorficznie. W takim razie $S^n$ jest uzwarceniem $S^n \backslash (0, 0, ... 1)$.

\item \textbf{Przestrzeń $(\mathbb{R}^n)^{+}$ jest uzwarceniem przestrzeni $\mathbb{R}^n$}\\
Przestrzeń $(\mathbb{R}^n)^{+}$ jest zwarta: załóżmy, że mamy pokrycie zbiorami ${U_i}_{i \in I}$. Wybierzmy spośród nich taki, który zawiera punkt w nieskończoności. Wtedy pozostałe z nich są pokryciem zwartej podprzestrzeni $\mathbb{R}^n$ (z definicji $(\mathbb{R}^n)^{+}$), więc można z nich wybrać podpokrycie skończone.
Dodatkowo $\mathbb{R}^n$ jest gęsta w $(\mathbb{R}^n)^{+}$, bo jeśli dopełnienie każdego zbióru otwartego zawierającego punkt w nieskończoności jest zwarte, to w szczególności nie jest całą przestrzenią $\mathbb{R}^n$, więc każdy taki zbiór ma z nią niepuste przecięcie.
$\mathbb{R}^n$ jest zanurzony homeomorficznie w $(\mathbb{R}^n)^{+}$, bo topologia w $(\mathbb{R}^n)^{+}$ zawiera topologię $\mathbb{R}^n$.
W takim razie $(\mathbb{R}^n)^{+}$ jest jednopunktowym uzwarceniem $\mathbb{R}^n$.
\end{itemize}

$ S^n \backslash \{(0, ... 0, 1)\}$ jest więc homeo z $\mathbb{R}^n$, więc z twierdzenia ich jednopunktowe uzwarcenia są homeomorficzne.
