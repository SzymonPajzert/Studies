\newpage
\section*{3 - 4}
Homeomorfizm pomiędzy
\begin{itemize}
  \item[3)] $(\R^n)^+$ - sferą Riemanna
  \item[4)] $D ^ {n} / \sim $ gdzie $ x \sim y \iff x = y $ lub $ x, y \in S^{n-1} $
\end{itemize}
Łatwo możemy obrać homeomorficzne przekształcenie z $D^n \setminus S^{n-1}$ do $R^n$ - rozciągając punkty na odcinku $[0, 1)$ w $[0, +\infty)$ dla każdej półprostej o początku w początku układu współrzędnych.\\
\\
Sfera Riemanna jest uzwarceniem $R^n$ a $D ^ {n} / \sim $ jest uzwarceniem $D^n \setminus S^{n-1}$. Pierwsze uzwarcenie wynika wprost z definicji. W drugim dopełnienie każdego zbioru otwartego zawierającego klasę abstrakcji brzegu jest zbiorem domkniętym. Ponieważ $D ^ {n} / \sim $ jest Hausdorffowskim obrazem zwartej przestrzeni w przekształceniu ilorazowym, to sama jest zwarta. Stąd dopełnienie każdego zbioru otwartego zawierającego brzeg jest zwarte, stąd $D ^ {n} / \sim $ jest rzeczywiście uzwarceniem. Ponieważ są to jednopunktowe uzwarcenia homeomorficznych lokalnie zwartych przestrzeni to na mocy Twierdzenia Aleksandrowa są one homeomorficzne.
