\newpage
\section*{1 - 6}
Homeomorfizm pomiędzy
\begin{itemize}
  \item[1)] Sferą $S^n$
  \item[6)] $A = (D^p \times S^q) \cup (S^{p-1} \times D^{q+1})$ gdzie $p+q = n$
\end{itemize}
% todo - dodaj strzałki nad x oraz y
Ustalmy dla $ z \in \mathbb{R}^{n+1}$ notację: $  z = (x(z), y(z)) = (x_1, ..., x_p, y_1, ..., y_{q+1}) $ gdzie jak łatwo zauważyć $x \in \mathbb{R}^p$ oraz $y \in \mathbb{R}^{q+1} $. \\
\\
Zauważmy, że dla $z \in A$ zachodzi:
\begin{align*}
  z \in A & \implies z \in (D^p \times S^q) \lor z \in (S^{p-1} \times D^{q+1}) \\
  z \in (D^p \times S^q) & \implies \sum x_i^2 \leq 1 \land \sum y_i^2 = 1 \\
  z \in (S^{p-1} \times D^{q+1}) & \implies \sum x_i^2 = 1 \land \sum y_i^2 \leq 1 \\
  z \in A & \implies 1 \leq ||z|| \leq \sqrt{2}
\end{align*}
Obierzmy teraz przekształcenie $ f : A \to S^n $ dane wzorem:
\begin{equation*}
  f((x_1, ..., x_p, y_1, ..., y_{q+1}) = z) = z \cdot \tfrac{1}{\sqrt{||x(z)||^2 + ||y(z)||^2}}
\end{equation*}
Ponieważ $ ||z|| = \sqrt{||x(z)||^2 + ||y(z)||^2} $\\
\\
Łatwo znależć dla niego przekształcenie odwrotne $ g : S^n \to A $ dane wzorem:
\begin{equation*}
  g((x(z), y(z)) = z) = z \cdot \tfrac{1}{\rm max(||x(z)||, ||y(z)||)}
\end{equation*}
\\
Przekształcenia $f$ oraz $g$ są wzajemnie odwrotne i przekształcają punkty z jednego zbioru na wspóliniowe z nimi i początkiem układu współrzędnych punkty drugiego zbioru. Ciągłość tych przekształceń wynika z ciągłości przekształceń modyfikujących długość wektora.
